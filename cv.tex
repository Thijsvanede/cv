% !BIB program = biber

% Create CV on A4 paper
\documentclass{cv}

\usepackage{amsmath}

% Name
\name{Thijs van Ede}
\subtitle{PhD Candidate at the University of Twente}

% Contact info - work
%---------------------------------------
\organization{University of Twente}
\department{Faculty of EEMCS}
\room{Zilverling 2042}

% Contact info location
%---------------------------------------
\address{P.O. Box 217}
\zipcode{7500 AE}
\cityname{Enschede}
\country{The Netherlands}

% Contact info social
%---------------------------------------
% Define email
\email{t.s.vanede@utwente.nl}
% Define google scholar page
\scholar{https://scholar.google.com/citations?user=tb8\_7jEAAAAJ\&hl=en\&oi=ao}
% Define github username
\github{Thijsvanede}
% Define linkedin shortlink
\linkedin{/in/thijsvanede}
% Define website
\website{https://vm-thijs.ewi.utwente.nl}

% Bibliography to use as publications
\addbibresource{bibliography.bib}

\begin{document}
	
	% Contact details
	\contact		
	
	% Research interests
	\section{Research Interests}
		My current research focuses on the detection of intrusions in evolving systems.
		Within this research, I concentrate on network-based approaches for constructing and analysing detection mechanisms.
		This involves modelling normal behaviour of systems, detecting anomalous behaviour, and most importantly, assessing the cause of these anomalies.
		By doing so, I intend to create an intrusion detection system that automatically evolves and deals with changes that are introduced to the monitored systems.
	
	% Educational background
	\section{Education}
		\cvitem{since 2018}{University of Twente}{PhD in Computer Science}{On Evolutionary Intrusion Detection for Changing Environments}
		\cvitem{2015-2017}{University of Twente}{Master Degree in Cyber Security - \emph{cum laude}}{Thesis: \link{https://essay.utwente.nl/74240/1/van\%20Ede_MA_EEMCS.pdf}{Detecting Adaptive Data Exfiltration in HTTP Traffic}}
		\cvitem{2012-2015}{University of Twente}{Bachelor Degree in Computer Science - \emph{with honours}}{Minor in Computer Science and Finance at \emph{M{\"a}lardalen University.}}
		\cvitem{2006-2012}{Chr. College Nassau Veluwe}{High School}{Pre-university education, specialising in science and engineering.}

	\section{Publications}
		% Ensure everything is cited
		\nocite{*}
		% Print bibliography
		\printbibliography[heading=none,title=none]
		
	\section{Teaching}
		\cvitemshort{since 2019}{University of Twente}{Teaching assistant for ``Privacy Enhancing Technologies''}
		\cvitemshort{since 2018}{University of Twente}{Teaching assistant for ``AI in Security''}
		
	\section{Advising}
		\cvitemshort{since 2019}{University of Twente}{\small Context Aware Identifier Deobfuscation for Android Applications \normalsize}
		\cvitemshort{\hfill 2019}{University of Twente}{\small \link{https://essay.utwente.nl/79732/1/TychoTeesselink_MA_EEMCS.pdf}{Identifying Application Phases in Mobile Encrypted Network Traffic} \normalsize}
		
	\section{Awards}
		\cvitemshort{\hfill 2018}{University of Twente}{Winner \link{https://www.utwente.nl/en/education/tgs/prospective-candidates/phd/for-ut-masters-students/bridging-fund/}{TGS bridging fund} and finalist of \link{https://www.utwente.nl/en/education/tgs/prospective-candidates/phd/for-ut-masters-students/bridging-fund/}{TGS Award}}
		\cvitemshort{\hfill 2018}{KHMW}{$3^\text{rd}$ place \link{https://www.khmw.nl/wp-content/uploads/Juryrapport-KNVI-Scriptieprijs-voor-Informatica-en-Informatiekunde-2018.pdf}{\small KNVI Scriptieprijs voor Informatica en Informatiekunde \normalsize}\\ for best M.Sc. Thesis in computer science}
		\cvitemshort{\hfill 2018}{PvIB}{$2^\text{nd}$ place \link{https://www.jbisa.nl/}{Joop Bautz Information Security Award}\\for best M.Sc. Thesis in cyber security}
		
	\section{References}
		\reference{Maarten van Steen}{Professor at University of Twente}{\link{mailto:m.r.vansteen@utwente.nl}{m.r.vansteen@utwente.nl}}
		\reference{Andreas Peter}{Associate Professor at University of Twente}{\link{mailto:a.peter@utwente.nl}{a.peter@utwente.nl}}
	
\end{document}
